\section{Conclusion}
% recapitulation of results and what we learned


The spectrum of the Earth as it would appear to an extrasolar observer is useful
for learning how to analyze the spectra of extrasolar planets. It 
illustrates both atmospheric and surface reflectivity features. 

The main contribuition for the spectra comes from the atmosphere transmission.
We also see enhanced reflectivity at short wavelengths from Rayleigh scattering
and
apparently negligible contributions from aerosol and ocean water scattering.  We
 see enhanced reflectivity of $4\pm 5\%$ at long wavelengths starting at
about
740 nm,
corresponding to the well-known vegetation red-edge. Our fittings for the
Earth's reflection spectrum shows good agreement with the combination of
reflectance, scattering, and transmission models.

The vegetation signal was not very 
significant mostly because of  the difficulties of the observation - we had a
signal-to-background ratios of  S/N $<$ 5. Although the observations
had great part of reflecting land  and had not much high
clouds in the atmosphere (see figures \ref{sat} and \ref{frommoon}), the low
signal is due to the fact that we only had 20 minutes of net observation for the
Earthshine (less than one fourth of the minimum we had estimated in
the beginning of the report).


In conclusion, we have shown that an observer in a nearby stellar
system, with the same or better resources used in this experiment, would be able
to use the oxygen, water and ozone absorption features to suspect the presence
of life on
Earth. The small chlorophyll
red-edge reflection feature might also help to confirm the presence of life.
