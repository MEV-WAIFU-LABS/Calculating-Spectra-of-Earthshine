\section{Observations}

This experiment aims to obtain an optical spectrum of Earthshine and test for
the presence of biomarker features,\eg O$_2$, O$_3$, H$_2$O, and the presence of
the vegetation red edge through fits of several models' components to the
observed spectrum. The observations were performed on the {\bf Mt. Stony Brook
14-inch telescope}, with the {\bf DADOS optical spectrograph} \cite{dados}, and
the {\bf SBIG STL-402 CCD camera} \cite{CCD}, on the date of April 17th, from
1:30 am to 5:30 am. The sky was clean and the temperature was around
55$^{\circ}$ F. The overall cloud cover from the archival satellite imagery and
the Earth seen from the moon, can be seen in the figures \ref{sat} and
\ref{frommoon},  in the appendix.




\subsection{Setting up the Spectrometer}
A spectrometer  splits the incoming light
of Earthshine by wavelength, having previously been focused by the optical
section of the instrument. The intensity of light at each wavelength is
measured and recorded by the detector, and it is then  plotted  against light
intensity, which is analyzed to find a number of
features. In order to pick up the key features of oxygen, water and vegetation
red edge, we investigate the visible and near infrared sections of
the electromagnetic spectrum, with light of wavelengths between  500 to
800 nm.






Earthshine is brightest when most of the Earth as seen from the Moon is
illuminated, {\it i.e.}, when the Moon is only a thin crescent. However, if the
Moon is too close to the Sun, there are  difficulties  separating
the glow of Earthshine from the bright twilight sky \cite{woolf_etal02}.
We select  a  night of waxing crescent moon when  the angular separation from
the Sun to the moon (from the moonrise to the sunrise) is suitable for the
experiment. This means that   the moon is sufficiently high above the
horizon ($>20^{\circ}$), yet sufficiently close to the Sun
($<90^{\circ}$) so that the Earthshine signal is bright.  The observations
cease at sunrise, since when the Sun reaches about 5$^{\circ}$ below the
horizon, the sky will  be too bright  to measure Earthshine. 



We set the optical wavelength range of the spectrograph to cover our chosen set
of Earthshine absorption by molecular O$_2$, O$_3$, H$_2$O. This was done before
mounting the spectrograph on the telescope with the help of the Neon light
source. We looked up the  wavelengths of the strongest Neon gas transitions
in the optical, adjusting the wavelength range of the spectrograph. We use the
DADOS spectrograph with long integration times. To maximize the spectral
resolution the narrowest width slit $ 25 \mu m$ is to be used. This gives a
spectral resolution of $\lambda / \Delta \lambda \sim 500$.


Setting  the telescope tracking rate to lunar ({\texttt Autostar II}
keypad), we obtain sequences of spectra of the {\bf bright} (moonshine) and {\bf dark} (earthshine) sides
of the facing Moon, each of them together with  the adjacent {\bf sky} (in the
adjacent slit). We also take
calibration exposures of the Neon light source and  darks with duration matched
to the duration of all of the science exposures.

\subsection{Estimative of Exposure Times}

The Earthshine  usually has low S/N (signal to background) ratio because the
observations are obtained with Moon low above the horizon, and consequently a
high airmass and low Earthshine fluxes with respect to the sky. Moreover, the
detector can be quickly saturated when recording the spectrum of the sunlit Moon
crescent. When it comes to the vegetation signal, the Earthshine data reduction
becomes even more difficult: past works \cite{seager_etal05} have
shown that it  is only a few percent (less than $5\%$) above
the continuum. Two reasons for this are pointed in the literature: (i) the variable amplitude, induced by a variable cloud cover and the Earth phase, (ii) the strong astmospheric bands which need to be removed to access the surface reflectance. 

Assuming that the signal is limited to photons, the signal to background ratio
for each of the sciences can be written as
\begin{equation}
 S/N = \sqrt{N_{\gamma}t},
\end{equation}
where $t$ is the time of integration ans $N_{\gamma}$ the number of photon
counts.

First, we make an estimate of the exposure time for the dark and
the bright side of the waxing crescent moon, supposing that their S/N are
similar. Considering that the {\it full moon} has magnitude
$m_{full} = -12.7$ and the {\it new moon} has magnitude $m_{new}= -2.5$, the
ratio of {\it photon fluxes} for the dark and bright side of the moon \cite{CCD}
is
\begin{eqnarray}
 R_{d/b} &\sim&\frac{F_{dark}}{F_{bright}}, \nonumber \\ 
&\sim& 0.1 \times 10^{(m_{full} - m_{new}) / 2.5}, \nonumber \\
&\sim& 10^{3}. \nonumber 
\end{eqnarray}

We can then derive the exposures times for each side,
\begin{eqnarray}
 t_{bright} \times N_{\gamma \, bright} &\sim& t_{dark} \times N_{\gamma \,
dark} \times R^{-1}_{d/b}.
\nonumber
\end{eqnarray}

The last result means that if we keep the number of counts constant for the dark
and bright sides, we should aim for calibration dark exposure times of
around 1000 
larger than those for bright, \eg we should have at least 90 minutes of net
observation of the dark side for a 5 seconds exposure of the bright side.
However, due the observational constrains, we ended up collecting only 1/4 of
this value, as it is shown in the Table \ref{looo} of the
appendix.




\subsection{Steps of Data Acquisition}

The data acquisition was  performed by the following steps, based on the
exposure times in the Table \ref{looo} in the appendix:
\begin{enumerate}
\item We record the spectrum of the Neon lamp on all the slits.

 \item We record a non-saturated spectrum of a {\it  bright stellar point
source} (Altair) at {\it two distinct positions on the slit}. We trace their
spectrum, which gives the direction along which we extract the spectrum of  the
Earthshine.

\item We track the {\it dark side of the Moon},  positioning the {\it dark limb}
in one slit and the adjacent on the other and recording the spectra.

\item We measure  the {\it illuminated limb of the moon } in the same way,
with a much smaller time to avoid saturation.

\item We repeat the above three procedures while the Sun is still not close
enough to the Moon.
\item   We obtain sets of dark frames for all the  exposure times of
our sciences.
\end{enumerate}










